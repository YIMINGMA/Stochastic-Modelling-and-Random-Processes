\section{Generators as Operators}

\subsection{Generators of Discrete Continuous-Time Markov Chains}

For a continuous-time Markov chain with a countable state space $S$, for any function $f: S \mapsto \mbb{R}$, we have 

\begin{equation*}
    \E [f(X(t))] = \sum_{x \in S} \mbf{\pi}_t(x) f(x) = \mbf{\pi}_t \vec{f},
\end{equation*}

where $\vec{f}$ is a column vector of values of $f$ at all the state $x \in S$.

We may be interested in how fast $\E [f(X(t))]$ varies with time $t$, so 

\begin{equation*}
    \frac{\dif}{\dif t} \E [f(X(t))] = \frac{\dif}{\dif t}\mbf{\pi}_t \vec{f} = \mbf{\pi}_t G \vec{f}.
\end{equation*}

Thus, we can think of the generator $G$ as acting on the function $f$ by 

\begin{equation*}
    (Gf) (x) = \sum_{y \in S} G_{x, y} f(y) = \sum_{\substack{y \neq x \\ y \in S}} G_{x, y} (f(y) - f(x)).
\end{equation*}

\subsection{Generators of Continuous Continuous-Time Markov Chains}

The idea of generators as operators can be extended to $S = \mbb{R}$ by replacing matrices and vectors with operators and functions.

\subsubsection{Generators of Brownian Motions}
 For a Brownian motion,

\begin{align*}
    \frac{\dif}{\dif t} \E [f(X(t))] = & \frac{\partial}{\partial t} \int_\mbb{R} p_t(x, y) f(y) \dif y \\ 
    = & \int_\mbb{R} \frac{\partial}{\partial t} p_t(x, y) f(y) \dif y \\ 
    = & \frac{1}{2} \int_\mbb{R} \frac{\partial^2}{\partial y^2} p_t(x, y) f(y) \dif y \\ 
    = & \E [(\mathcal{L}f)(X(t))]
\end{align*}

with $(\mathcal{L}f)(x) := \frac{1}{2} f''(x)$, assuming $f$ is twice differentiable and $f(x) \, \& f'(x) \to 0$ as $x \to \pm \infty$ (integration by parts). $\mathcal{L}$ is the generator but now a linear operator on functions.

\subsubsection{Generators of Jump Processes}

For a jump process on $\mbb{R}$, 

\begin{equation*}
    (\mathcal{L}f)(x) = \int_\mbb{R} r(x,y) [f(y) - f(x)] \dif y.
\end{equation*}

We can obtain the Brownian motion as a scaling limit of a jump process. Take a jump process $X(t)$ with $r(x, y) = q(y-x)$ such that $\int_\mbb{R} z q(z) \dif z = 0$ and $\int_\mbb{R} z^2 q(z) \dif z = \sigma^2 \in (0, \infty)$. Then $\forall T > 0$, with $X(0) = 0$,

\begin{equation*}
    \left. \frac{\epsilon}{\sigma} X(\frac{t}{\epsilon^2}) \right|_{t \in [0, T]} \xrightarrow{\text{d}} \left. B(t) \right|_{t \in [0, T]}, \; \text{as} \; \epsilon \to 0.
\end{equation*}

We can prove this by Taylor expansion of the generator 

\begin{equation*}
    f(y) = f(x) + (y-x) f'(x) + \frac{1}{2} (y-x)^2 f''(x) + \cdots,
\end{equation*}

and tightness of the set $S$ of probability distributions for the scaled jump process: $\forall \eta > 0$, $\exists K \in \mbb{R}$ such that for all $\mu \in \bar{S}$, $\mu (K^c) < \eta$.
