\section{Introduction}

Any realistic model of a real-world phenomenon must take into account the possibility of randomness. That is, more often than not, the quantities we are interested in will not be predictable in advance but, rather, will exhibit an inherent variation that should be taken into account by the model. This is usually accomplished by allowing the model to be probabilistic in nature. Such a model is, naturally enough, referred to as a probability model.

The majority of the chapters of this book will be concerend with different probability models of natural phenomena. Clearly, in order to master both the ``model building'' and the subsequent analysis of these models, we must have a certain knowledge of basic probability theory. The remainder of this chapter, as well as the next two chapters, will be concerned with a study of this subject.