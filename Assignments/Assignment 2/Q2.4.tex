\section{Birth-Death Process}


A birth-death process $(X_t: t \ge 0)$ is a continuous-time Markov chain with state space $S = \mbb{N}_0 = \{0, 1, \cdots\}$ and jump rates 
$$
x \xrightarrow{\alpha_x} x + 1 \, \text{for all $x \in S$, } x \xrightarrow{\beta_x} x - 1 \, \text{for all $x \ge 1$.}
$$

According to the article \cite{karlin1957classification} of Samuel Karlin and James McGregor, a sufficient and necessary condition for the states of a birth-death process being recurrent is 
\begin{equation}\label{recurrent}
    \sum_{i=1}^\infty \prod_{n=1}^i \frac{\beta_n}{\alpha_n} = \infty,
\end{equation}
sufficient and necessary conditions for them being null recurrent is 
\begin{equation}\label{null recurrent}
    \sum_{i=1}^\infty \prod_{n=1}^i \frac{\beta_n}{\alpha_n} = \infty \quad \text{and} \quad
    \sum_{i=1}^\infty \prod_{n=1}^i \frac{\alpha_{n-1}}{\beta_n} = \infty,
\end{equation}
and sufficient and necessary conditions for them being ergodic is 
\begin{equation}\label{ergodic}
    \sum_{i=1}^\infty \prod_{n=1}^i \frac{\beta_n}{\alpha_n} = \infty \quad \text{and} \quad
    \sum_{i=1}^\infty \prod_{n=1}^i \frac{\alpha_{n-1}}{\beta_n} < \infty,
\end{equation}

\begin{enumerate}
    \item[(a)] Suppose $\alpha_x = \alpha > 0$ for $x \ge 0$ and $\beta_x = \beta > 0$ for $x > 0$. Consider different cases depending on the choice of $\alpha$ and $\beta$ where necessary:
    \begin{itemize}
        \item Is $(X_t: t \ge 0)$ irreducible? Give all communicating classes in $\mbb{N}_0$ and state whether they are transient or null/positive recurrent.
        
        \textit{ Sol. } Since all states are accessible from each other, with positive rates $\alpha$ and $\beta$, all states in $S$ are communicating states. Thus, the process is irreducible.

        $\forall i \ge 1$, $\prod_{n=1}^i \frac{\beta_n}{\alpha_n} = \prod_{n=1}^i \frac{\beta}{\alpha} = \left( \frac{\beta}{\alpha} \right)^i$, so 
        \begin{equation*}
            \sum_{i=1}^\infty \prod_{n=1}^i \frac{\beta_n}{\alpha_n} = \sum_{i=1}^\infty \left( \frac{\beta}{\alpha} \right)^i.
        \end{equation*}
        According to \eqref{recurrent}, these states are recurrent if and only if 
        \begin{equation*}
            \sum_{i=1}^\infty \left( \frac{\beta}{\alpha} \right)^i = \infty,
        \end{equation*}
        which is equivalent to 
        \begin{equation*}
            \frac{\beta}{\alpha} \ge 1.
        \end{equation*}
        Thus, the states are transient if and only if 
        \begin{equation*}
            0 < \frac{\beta}{\alpha} < 1.
        \end{equation*}
        And similarly, by \eqref{null recurrent}, the states are null recurrent if and only if 
        \begin{equation*}
            \frac{\beta}{\alpha} = 1,
        \end{equation*}
        and they are positive recurrent if and only if 
        \begin{equation*}
            \frac{\beta}{\alpha} > 1.
        \end{equation*}
        \item Give all stationary distributions and state whether they are reversible.
        
        \textit{ Sol. } We can use the $\langle \mbf{\pi} | G = \langle \mbf{0} |$ 
        to solve stationary distributions. Notice that 
        \begin{equation*}
            G = 
            \begin{bmatrix}
                - \alpha & \alpha & 0 & 0 & 0 & \cdots \\ 
                \beta & -(\alpha+\beta) & \alpha & 0 & 0 & \cdots \\ 
                0 & \beta & -(\alpha+\beta) & \alpha & 0 & \cdots \\ 
                0 & 0 & \beta & -(\alpha+\beta) & \alpha & \cdots \\ 
                0 & 0 & 0 & \beta & - (\alpha+\beta) & \cdots \\ 
                \vdots & \vdots & \vdots & \vdots & \vdots &  \ddots \\
            \end{bmatrix},
        \end{equation*}
        so 
        \begin{align}
            -\alpha \mbf{\pi}_0 + \beta \mbf{\pi}_1 & = 0 \label{eqn25}\\ 
            \alpha \mbf{\pi}_0 - (\alpha + \beta) \mbf{\pi}_1 + \beta \mbf{\pi}_2 &= 0 \label{eqn26}\\ 
            \alpha \mbf{\pi}_1 - (\alpha + \beta) \mbf{\pi}_2 + \beta \mbf{\pi}_3 & = 0 \label{eqn27}\\ 
            & \vdots \notag
        \end{align}
        To solve these equations, notice that \eqref{eqn25} gives
        \begin{equation}\label{eqn28}
            \beta \mbf{\pi}_1 = \alpha \mbf{\pi}_0.
        \end{equation}
        Plug \eqref{eqn28} into \eqref{eqn26}, and we get 
        \begin{equation}\label{eqn29}
            \beta \mbf{\pi}_2 = \alpha \mbf{\pi}_1.
        \end{equation}
        And pluging \eqref{eqn29} into \eqref{eqn27} results in 
        \begin{equation}
            \beta \mbf{\pi}_3 = \alpha \mbf{\pi}_2.
        \end{equation}
        So we can keep doing this recursively and get 
        \begin{equation*}
            \beta \mbf{\pi}_{n+1} = \alpha \mbf{\pi}_{n}, \quad \forall n \in \mbb{N}_0,
        \end{equation*}
        or equivalently,
        \begin{equation*}
            \mbf{\pi}_n = \left( \frac{\alpha}{\beta} \right)^n \mbf{\pi}_0.
        \end{equation*}
        If $\frac{\alpha}{\beta} \ge 1$, there is no stationary distribution, because 
        \begin{equation*}
            \sum_{n = 0}^\infty \mbf{\pi}_n = \sum_{n = 0}^\infty \left( \frac{\alpha}{\beta} \right)^n \mbf{\pi}_0 \neq 1.
        \end{equation*}
        If $\frac{\alpha}{\beta} < 1$, then solving $\sum_{n = 0}^\infty \mbf{\pi}_n = 1$ gives 
        \begin{equation*}
            \mbf{\pi}_0 = \frac{\beta - \alpha}{\beta},
        \end{equation*}
        so 
        \begin{equation*}
            \mbf{\pi}_n = \left( \frac{\alpha}{\beta} \right)^n \cdot \frac{\beta - \alpha}{\beta}.
        \end{equation*}
        Since $G_{x,y} \mbf{\pi}_x = G_{y,x} \mbf{\pi}_y$ for all $x,y \in S$, the process is time reversible.

        \item Is the process ergodic?
        
        \textit{ Sol. } To check the ergodicity of the process, we only need to verify \eqref{ergodic}.
        \begin{align*}
            \infty = & \sum_{i=1}^\infty \prod_{n=1}^i \frac{\beta_n}{\alpha_n} \\ 
            = & \sum_{i = 1}^\infty \prod_{n=1}^i \frac{\beta}{\alpha} \\ 
            = & \sum_{i = 1}^\infty \left( \frac{\beta}{\alpha} \right)^i,
        \end{align*}
        which holds only when 
        \begin{equation}\label{eqn30}
            \frac{\beta}{\alpha}  \ge 1.
        \end{equation}
        \begin{align*}
            \infty > & \sum_{i=1}^\infty \prod_{n=1}^i \frac{\alpha_{n-1}}{\beta_n}\\ 
            = & \sum_{i=1}^\infty \prod_{n=1}^i \frac{\alpha}{\beta} \\ 
            = & \sum_{i=1}^\infty \left(\frac{\alpha}{\beta} \right)^i
        \end{align*}
        which can be true only when 
        \begin{equation*}\label{eqn31}
            \frac{\alpha}{\beta} < 1
        \end{equation*}
        Based on \eqref{eqn30} and \eqref{eqn31}, the process is ergodic if and only if 
        \begin{equation*}
            \frac{\beta}{\alpha} > 1.
        \end{equation*}
    \end{itemize}

    \item[(b)] Suppose $\alpha_x = x \alpha$, $\beta_x = x \beta$ for $x \ge 0$ with $\alpha,\beta > 0$ and $X_0 = 1$. Consider different cases depending on the choice of $\alpha$ and $\beta$ where necessary: 
    \begin{itemize}
        \item Is $(X_t: t \ge 0)$ irreducible? Give all communicating classes in $\mbb{N}_0$ and state whether they are transient or null/positive recurrent.
        
        \textit{ Sol. } Notice that $\alpha_0 = 0$ and $\beta_0 = \beta > 0$, so state $0$ is absorbing, and thus positive recurrent. For other states, they are communicating. So state space can be decomposed into $\{0\}$ and $\mbb{N}_+$, which means the process is not irreducible.

        Since
        \begin{align*}
            \sum_{i=1}^\infty \prod_{n=1}^i \frac{\beta_n}{\alpha_n} = & \sum_{i=1}^\infty \prod_{n=1}^i \frac{\beta n}{\alpha n} \\
            = & \sum_{i=1}^\infty \prod_{n=1}^i \frac{\beta}{\alpha} \\ 
            = & \sum_{i=1}^\infty \left( \frac{\beta}{\alpha} \right)^i,
        \end{align*}
        by \eqref{recurrent}, states $1, 2, \cdots$ are transient if and only if $\sum_{i=1}^\infty \prod_{n=1}^i \frac{\beta_n}{\alpha_n} < \infty$, which is equivalent to $\frac{\beta}{\alpha} < 1$. 

        Now suppoes $\frac{\beta}{\alpha} \ge 1$ so that states $1, 2, \cdots$ are  recurrent. 
        Since 
        \begin{equation*}
            \sum_{i=1}^\infty \prod_{n=1}^i \frac{\alpha_{n-1}}{\beta_n} = \sum_{i=1}^\infty \prod_{n=1}^i \frac{\alpha(n-1)}{\beta n} = 0 < \infty,
        \end{equation*}
        by \eqref{null recurrent}, states $1, 2, \cdots$ positive recurrent. 

        \item Give all stationary distributions and state whether they are reversible.
        
        \textit{ Sol. } The transition matrix $G$ is
        \begin{equation*}
            \begin{bmatrix}
                0 & 0 & 0 & 0 & 0 & \cdots \\ 
                \beta & -(\alpha + \beta) & \alpha & 0 & 0 & \cdots \\ 
                0 & 2\beta & - 2(\alpha + \beta) & 2\alpha & 0 & \cdots \\ 
                0 & 0 & 3\beta & -3(\alpha + \beta) & 3\alpha & \cdots \\ 
                0 & 0 & 0 & 4\beta & -4(\alpha + \beta) & \cdots \\ 
                \vdots & \vdots & \vdots & \vdots & \vdots & \ddots  
            \end{bmatrix}
        \end{equation*}
        $\langle \mbf{\pi} | G = \langle \mbf{0} |$ gives 
        \begin{align}
            \beta\mbf{\pi}_1 & = 0 \label{eqn32}\\ 
            -(\alpha + \beta) \mbf{\pi}_1 + 2 \beta \mbf{\pi}_2 & = 0 \label{eqn33}\\ 
            \alpha \mbf{\pi}_1 - 2(\alpha + \beta) \mbf{\pi}_2 + 3 \beta \mbf{\pi}_3 & = 0 \label{eqn34}\\ 
            2\alpha \mbf{\pi}_2 - 3(\alpha + \beta) \mbf{\pi}_3 + 4 \beta \mbf{\pi}_4 & = 0 \label{eqn35}\\ 
            & \vdots \notag
        \end{align}
        Solve \eqref{eqn32} first, and this gives $\mbf{\pi}_1 = 0$. Using ``forward-substitution'' in \eqref{eqn33}, \eqref{eqn34}, \eqref{eqn35}, ... gives $\mbf{\pi}_2 = \mbf{\pi}_3 = \mbf{\pi}_4 = \cdots = 0$. Thus, the only stationary distribution is $\mbf{e}_1$, which is 
        \begin{equation*}
            [1, \, 0, \, 0, \, 0, \, \cdots].
        \end{equation*}
        Once being absorbed into state $0$, the process cannot go back into other states, so the process is not time reversible.

        \item Derive an equation for $\mu_t = \mbb{E}[X_t]$ and solve it for initial condition $\mu_0 = 1$.
        
        \textit{ Sol. } For any function $f: S \mapsto \mbb{R}$, 
        \begin{align*}
            (\mathcal{L}f)(x) = & G | f \rangle (x) \\ 
            = & \sum_{y \neq x} G_{x,y} [f(y) - f(x)] \\ 
            = & x \beta [f(x-1) - f(x)] + x \alpha [f(x+1) - f(x)] \\ 
        \end{align*}
        So 
        \begin{align*}
            \frac{\dif}{\dif t} \mbb{E} [f(X_t)] = & \mbb{E} [(\mathcal{L}f)(X_t)] \\ 
            = & \mbb{E} [X_t\beta [f(X_t - 1) - f(X_t)] + X_t \alpha [f(X_t + 1) - f(X_t)]].
        \end{align*}
        Let $f(x) = x$, then we have 
        \begin{align*}
            \frac{\dif}{\dif t} \mbb{E} [X_t] = & \mbb{E} [X_t \beta ( X_t - 1 - X_t) + X_t \alpha (X_t + 1 - X_t)] \\ 
            = & \mbb{E} [- \beta X_t + \alpha X_t] \\ 
            = & (\alpha - \beta)\mbb{E} [X_t], 
        \end{align*}
        whose general solution is give by 
        \begin{equation}\label{eqn36}
            \mbb{E} [X_t] = C e^{(\alpha - \beta) t}.
        \end{equation}
        By the intial condition $\mbb{E} [X_0] = \mu_0 =1$, we know $C = 1$ in \eqref{eqn36}. So the solution is 
        \begin{equation*}
            \mu_t = e^{(\alpha - \beta) t}.
        \end{equation*}

        \item Set up a recursion for the ``extinction probability'' $h_x = \mbb{P}[\lim\limits_{t\to\infty}X_t = 0 | X_0 = x]$ and give the smallest solution with boundary condition $h_0 = 1$.
        
        \textit{ Sol. }
        Since we consider the limit of $X_t$ as $t \to \infty$, it is reasonable to assume $t$ is large enough so that jumps can happen. Obviously, we have $h_0 = 0$, so now assume $x > 0$. 
        
        Let $J_1 = \inf\{t > 0: X_t \neq X_0 \}$ be the time of the first jump, so 
        \begin{equation*}
            \mbb{P}[X_{J_1} = x + 1 | X_0 = x] = \frac{G_{x,x+1}}{|G_{x,x}|} = \frac{x \alpha}{x (\alpha + \beta)} = \frac{\alpha}{\alpha + \beta},
        \end{equation*}
        and 
        \begin{equation*}
            \mbb{P}[X_{J_1} = x - 1 | X_0 = x] = \frac{G_{x,x-1}}{|G_{x,x}|} = \frac{x \beta}{x (\alpha + \beta)} = \frac{\beta}{\alpha + \beta}.
        \end{equation*}
        Now condition on $X_{J_1}$, we have 
        \begin{align}\label{eqn37}
            h_x = & \mbb{P} [\lim_{t\to\infty} X_t = 0 | X_0 = x] \notag \\ 
            = & \mbb{P} [\lim_{t\to\infty} X_t = 0 | X_0 = x, X_{J_1} = x+1] \cdot \mbb{P}[X_{J_1} = x + 1 | X_0 = x] \notag \\ 
            & + \mbb{P} [\lim_{t\to\infty} X_t = 0 | X_0 = x, X_{J_1} = x-1] \cdot \mbb{P}[X_{J_1} = x - 1 | X_0 = x] \notag \\ 
            = & \mbb{P} [\lim_{t\to\infty} X_t = 0 | X_{J_1} = x+1] \cdot \mbb{P}[X_{J_1} = x + 1 | X_0 = x] \notag \\ 
            & + \mbb{P} [\lim_{t\to\infty} X_t = 0 | X_{J_1} = x-1] \cdot \mbb{P}[X_{J_1} = x - 1 | X_0 = x] \quad \text{(by the Markov property)} \notag \\ 
            = & \mbb{P} [\lim_{t\to\infty} X_t = 0 | X_{0} = x+1] \cdot \mbb{P}[X_{J_1} = x + 1 | X_0 = x] \notag \\ 
            & + \mbb{P} [\lim_{t\to\infty} X_t = 0 | X_{0} = x-1] \cdot \mbb{P}[X_{J_1} = x - 1 | X_0 = x] \quad \text{(by homogeneity)} \notag \\ 
            = & h_{x+1} \cdot \frac{\alpha}{\alpha + \beta}  +  h_{x-1} \cdot \frac{\beta}{\alpha + \beta}
        \end{align}
        The characteristic equation of \eqref{eqn37} is 
        \begin{align*}
            t^2 \cdot \frac{\alpha}{\alpha + \beta} + \frac{\beta}{\alpha + \beta} & = t \\ 
            \alpha t^2 - (\alpha + \beta) t + \beta & = 0 \\ 
            (\alpha t - \beta) (t - 1) & = 0,
        \end{align*}
        whose roots are 
        \begin{equation*}
            t_1 = \frac{\beta}{\alpha} \quad \& \quad t_2 = 1.
        \end{equation*}
        If $\alpha = \beta$, then $t_1 = t_2 = 1$, and the general solution to \eqref{eqn37} is 
        \begin{align*}
            h_n = & a \cdot 1^n + b \cdot n \cdot 1^2 \\ 
            = & a + bn,
        \end{align*}
        where $a,\,b\in\mbb{R}$ are two constants. By $h_0 = 1$, we have $a = 1$, so $h_x = 1 + bx$. But $h_x$ is a probability, which means it should range between $0$ and $1$, so we have $b = 0$. So in the case of $\alpha = \beta$, $h_x \equiv 1$.

        Now suppose $\alpha \neq \beta$, so $t_1 \neq t_2$. Then the general solution to \eqref{eqn37} is 
        \begin{align*}
            h_n = & a \cdot 1^n + b \cdot \left( \frac{\beta}{\alpha} \right)^n \\ 
            = & a + b \left( \frac{\beta}{\alpha} \right)^n,
        \end{align*}
        where $a,\,b\in\mbb{R}$ are two constants. Again, use the boundary condition $h_0 = 1$, and we have $a = 1 - b$. Thus, 
        $$h_x = 1 - b + b \left( \frac{\beta}{\alpha} \right)^x.$$

        If $\beta > \alpha$, then $\frac{\beta}{\alpha} > 1$, and as a result, 
        $\lim_{x \to \infty} (\beta / \alpha)^n = \infty$. In order that $h_x$ is a probability, $b$ has to be $0$. So in this case $h_x \equiv 1$. 

        If $\beta < \alpha$, then $\frac{\beta}{\alpha} < 1$, and thus, $(\beta / \alpha)^x le 1$ for all $x \ge 0$. To ensure $h_x \le 1$, let
        \begin{align*}
             1 - b + b \left( \frac{\beta}{\alpha} \right)^x \le & 1 \\ 
             \left( \left( \frac{\beta}{\alpha} \right)^x - 1 \right) b \le & 0\\
             b \ge & 0.
        \end{align*}
        In order to make $h_x$ nonnegative, we need 
        \begin{align*}
            0 \le & 1 - b + b \left( \frac{\beta}{\alpha} \right)^x \\
            1 \ge & \left(1 - \left( \frac{\beta}{\alpha} \right)^x \right) b \\
            b \le & \frac{1}{1 - \left( \frac{\beta}{\alpha}\right)^x} \\ 
            b \le & 1.
        \end{align*}
        So the range of $b$ is $[0, 1]$.

        To conclude, the solution to \eqref{eqn37} is 
        \begin{equation*}
            h_x = \begin{cases}
                1 \qquad & \beta \ge \alpha > 0\\ 
                1 - b + b \left( \frac{\beta}{\alpha} \right)^x \qquad & 0 < \beta < \alpha
            \end{cases},
        \end{equation*}
        where $b \in [0, 1]$ is a constant.

        Obviously, we have made $1 - b + b \left( \frac{\beta}{\alpha} \right)^x \le 1$ when we determined the range of $b$, so the smaller solution is 
        $$
        1 - b + b \left( \frac{\beta}{\alpha} \right)^x,
        $$
        with $0 < \beta < \alpha$ and $b \in [0, 1]$. Let 
        $H(b) = 1 - b + b \left( \frac{\beta}{\alpha} \right)^x$, then 
        $$H'(b) = -1 + \left( \frac{\beta}{\alpha} \right)^x \le 0,$$
        which means $H(b)$ is monotonically decreasing in $[0, 1]$. So the smallest solution is reached when $b = 1$, and in this circumstance, 
        $$h_x = \left( \frac{\beta}{\alpha} \right)^x.$$

        \item Is the process ergodic?
        
        \textit{ Sol. } 
        By \eqref{ergodic}, the process is ergodic if and only if 
        \begin{equation*}
            \sum_{i=1}^\infty \prod_{n=1}^i \frac{\beta_n}{\alpha_n} = \infty \quad \text{and} \quad
            \sum_{i=1}^\infty \prod_{n=1}^i \frac{\alpha_{n-1}}{\beta_n} < \infty.
        \end{equation*}
        The first condition is 
        \begin{align*}
            \infty & = \sum_{i=1}^\infty \prod_{n=1}^i \frac{\beta_n}{\alpha_n} \\ 
            & = \sum_{i=1}^\infty \prod_{n=1}^i \frac{\beta n}{\alpha n} \\ 
            & = \sum_{i=1}^\infty \prod_{n=1}^i \frac{\beta}{\alpha} \\ 
            & = \sum_{i=1}^\infty \left( \frac{\beta}{\alpha} \right)^i,
        \end{align*}
        which is equivalent to $\beta \ge \alpha$. 

        As for the second condition, we have 
        \begin{equation*}
            \sum_{i=1}^\infty \prod_{n=1}^i \frac{\alpha_{n-1}}{\beta_n} = \sum_{i=1}^\infty \prod_{n=1}^i \frac{\alpha(n-1)}{\beta n} = 0 < \infty.
        \end{equation*}
        
        So the process is ergodic if and only if $\beta \ge \alpha$.
    \end{itemize} 
\end{enumerate}