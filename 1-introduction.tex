\chapter{Introduction}

\section{Motivation}

Suppose we are modelling COVID. Let 
\begin{itemize}
    \item \textcolor{myblue}{$S$ be he number of the susceptible;}
    \item \textcolor{myblue}{$I$ be the number of the infected;}
    \item \textcolor{myblue}{$R$ be the number of the removed (those who have either recovered or died).}
\end{itemize}

\subsection{A Deterministic Model}

\textcolor{myblue}{
    A deterministic model might be
    \begin{align*}
        \dot{S} = & - \beta I S, \\ 
        \dot{I} = & \beta I S - \gamma I, \\ 
        \dot{R} = & \gamma I.
    \end{align*}
}
\textcolor{myorange}{
    But there are some problems in this model:
    \begin{itemize}
        \item $S$, $I$ and $R$ are integers, so it does not make sense to talk about $\dot{S}$, $\dot{I}$ and $\dot{R}$. 
        \item There is variability in when contacts are made and lead to infection.
    \end{itemize}
}

\subsection{A Stochastic Model}

A better model might be \textcolor{myblue}{stochastic
    \begin{align*}
        \prob{S \to S-1 \; \& \; I \to I-1 \; \text{in} \; \Delta t} = &\beta I S \Delta t + o(\Delta t) \\ 
        \prob{I \to I-1 \; \& \; R \to R+1 \; \text{in} \; \Delta t} = & \gamma I \Delta t + o(\Delta t).
    \end{align*}
}
\textcolor{myorange}{The problem of this model is that contacts are usually not made uniformally in the whole population.}

\subsection{A Network Model}

\textcolor{myblue}{We can use a network model, in which nodes represent individuals and edge weights represent contact rates, to avoid uniform contacts. }\textcolor{myorange}{But tis is unrealistic: the network is too big to represent 60 million people in the UK.}

\subsection{A Random Network Model}

\textcolor{myblue}{Based on the network model, we can make probability distributions on networks and derive probabilistic conclusions over the combination of stochastic dynamics and randomness of networks.}
