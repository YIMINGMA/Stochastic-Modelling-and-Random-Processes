\section{Probability Theory}

Suppose we are doing an experiment which have different random outcomes.

\begin{definition}[Sample Spaces]
    The \textbf{sample space} of the experiment is the set of all possible outcomes, denoted as $\Omega$.
\end{definition}

\begin{definition}[Sigma Algebra]
    The $\mbf{\sigma}$\textbf{-algebra} of subsets of $\Omega$, denoted as $\mathcal{F}$, is a set of subsets of $\Omega$ which satisfies:
    \begin{itemize}
        \item $\Omega \in \mathcal{F}$;
        \item $A \in \mathcal{F} \implies A^c \in \mathcal{F}$;
        \item $\{ A_i | i \in \mathcal{I} \} \subset \mathcal{F}$ with $\mathcal{I}$ being countable $\implies \bigcup\limits_{i \in \mathcal{I}} A_i \in \mathcal{F}$.
    \end{itemize}
\end{definition}

\begin{remark}
    We say $\mathcal{I}$ is countable if there exists a one-to-one map from $\mathcal{I}$ into $\mbb{Z}$, so ``countable'' includes ``finite''.
\end{remark}

\begin{example}
    If $\Omega$ is countable, we usually take \textcolor{myblue}{$\mathcal{F} = 2^\Omega$}, which is the \textcolor{myblue}{power set of $\Omega$}.
\end{example}

\begin{example}
    When $\Omega$ is not countable, e.g. $[0,1]$, if you allow \href{https://en.wikipedia.org/wiki/Axiom_of_choice}{Axiom of Choice} \footnote{A Cartesian product of a collection of nonempty sets is nonempty.}, then there exist unmeasurable subsets, and we exclude them from $\mathcal{F}$, i.e. $\mathcal{F}$ is the set of all Lebesgue-measurable subsets on $[0,1]$.
\end{example}

\begin{definition}[Events]
    The members of $\mathcal{F}$ are called \textbf{events}.
\end{definition}

\begin{definition}[Probability]
    $\prob[\cdot]: \mathcal{F} \mapsto \mbb{R}$ is called a probability if
    \begin{itemize}
        \item $\prob[A] \in [0,1]$, $\forall A \in \mathcal{F}$;
        \item $\prob[\Omega] = 1$;
        \item $\prob[\cdot]$ satisfies the \textbf{countable additivity}: $\forall \{A_i | i \in \mathcal{I} \} \subset \mathcal{F}$, where $\mathcal{I}$ is a countable set, if $A_i$'s are disjoint, then 
        \begin{equation*}
            \prob \left[ \bigcup_{i \in \mathcal{I}} A_i \right] = \sum_{i \in \mathcal{I}} \prob \left[A_i\right].
        \end{equation*}
    \end{itemize}
\end{definition}

\begin{definition}[Independence]
    Say $A, B \in \mathcal{F}$ are \textbf{independent} if
    \begin{equation*}
        \prob \left[ A \cap B \right] = \prob [A] \prob [B].
    \end{equation*}
\end{definition}

\begin{definition}[Conditional Probabilities]
    If $\prob [B] > 0$, then the \textbf{conditional probability} $\prob \left[ A | B \right]$ is defined by 
    \begin{equation*}
        \prob \left[ A | B \right] = \frac{\prob [A \cap B]}{\prob [B]}, \quad \forall A \in \mathcal{F}.
    \end{equation*}
\end{definition}

\begin{definition}[Partitions]
    $\{ B_i | i \in \mathcal{I} \}$ is called a \textbf{partition} of the sample space $\Omega$ if:
    \begin{itemize}
        \item $B_i$'s are \textbf{pairwise disjoint}: $B_i \cap B_j = \emptyset$, $\forall i, j \in \mathcal{I}$, $i \neq j$;
        \item $B_i \neq \emptyset$, $\forall i \in \mathcal{I}$;
        \item $\{ B_i | i \in \mathcal{I} \}$ \textbf{covers} $\Omega$: $\bigcup_{i\in \mathcal{I}} B_i = \Omega$.
    \end{itemize}
\end{definition}

\begin{theorem}[The Law of Total Probability]
    Let $\{ B_i | i \in \mathcal{I} \}$ be a countable partition of $\Omega$, with $B_i \in \mathcal{F}$ and $\prob \left[ B_i \right] > 0$, $\forall i \in \mathcal{I}$. Then $\forall A \subset \mathcal{F}$, we have 
    \begin{equation*}
        \prob \left[ A \right] = \sum_{i \in \mathcal{I}} \prob \left[ A | B_i \right] \prob \left[ B_i \right].
    \end{equation*}
\end{theorem}

\begin{theorem}[Bayes' Rule]
    For any events $A$ and $B$, if $\prob [A] > 0$ and $\prob [B] > 0$, then 
    \begin{equation*}
        \prob \left[ B | A \right] = \frac{\prob \left[ A | B \right] \prob [B]}{\prob [A]}.
    \end{equation*}
    Furthermore, if $\{B_i | i \in \mathcal{I} \}$ is a countable partition of $\Omega$, with $B_i \in \mathcal{F}$ and $\prob \left[ B_i \right] > 0$, $\forall i \in \mathcal{I}$, then 
    \begin{equation*}
        \prob \left[ B_i | A \right] = \frac{\prob \left[ A | B_i \right] \prob[B_i]}{ \sum_{i\in\mathcal{I}} \prob \left[ A | B_i \right] \prob \left[ B_i \right]}.
    \end{equation*}
\end{theorem}

\begin{example}
    Suppose the \textcolor{myblue}{true positive rate} $\prob \left( \text{tests positive} | \text{has COVID}\right)$ is $0.99$ and the \textcolor{myblue}{false positive rate} $\prob \left( \text{tests positive} | \text{does not have COVID} \right)$ is $0.01$. Suppose in the population, the probability of getting contracted with COVID is $0.001$, i.e. $\prob \left( \text{has COVID} \right) = 0.001$, what is the probability that a person has COVID given his/her test is positive?

    \textit{ Sol. }
    \begin{eqnarray*}
        \lefteqn{\prob \left(\text{has COVID} | \text{tests positive} \right)} \\ 
        & = & \frac{ \prob \left( \text{tests positive} | \text{has COVID} \right) \prob \left( \text{has COVID} \right)}{\prob (\text{tests positive})} \\ 
        & = & \prob \left( \text{tests positive} | \text{has COVID} \right) \prob \left( \text{has COVID} \right) \\ 
        & & \div \left[ \prob \left(\text{tests positive} | \text{has COVID} \right) \prob \left(\text{has COVID}\right) \right. \\
        & & \left. + \prob \left(\text{tests positive} | \text{does not have COVID} \right) \prob \left(\text{does not have COVID}\right) \right] \\ 
        & = & \frac{0.99 \times 0.001}{0.99 \times 0.001 + 0.01 \times (1-0.001)} \\ 
        & \approx & 0.090.
    \end{eqnarray*}
\end{example}