\section{More on Generators}

The generator $\mathcal{L}$ are defined on functions on the state space but also tell you how probability distributions evolve, using the adjoint $\mathcal{L}^*$. 

Probability distributions are linear functionals on a set of continuous functions $S \mapsto \mbb{R}$, in comparison with row vectors in the case of discrete state space in which a row vector is a linear functional on a set of possible column vectors. Represent a linear functional when $S = \mbb{R}$ by an integral with respect to a probability density $p$:
\begin{equation*}
    f \mapsto \int_{\mbb{R}} p(x) f(x) \dif x \in \mbb{R}.
\end{equation*}

We start from 
\begin{equation*}
    \frac{\dif}{\dif t} \E [f(X(t))] = \E [f(X(t))].
\end{equation*}
Notice that 
\begin{equation*}
    \frac{\dif}{\dif t} \int_\mbb{R} p_t(x, y) f(y) \dif y =  \int_\mbb{R} p_t(x, y) \mathcal{L}f(y) \dif y.
\end{equation*}
Suppose we are considering the diffusion process with $\mathcal{L}(f) = af'+ \frac{1}{2} \sigma^2 f''$, and assume $p \, \& \frac{\partial p}{\partial y} \to 0$ as $y \to \infty$. Integrate by parts (twice) to get 
\begin{align*}
    \frac{\dif}{\dif t} \int_\mbb{R} p_t(x, y) f(y) \dif y = & \int_\mbb{R} p_t(x, y) \mathcal{L}f(y) \dif y \\ 
    = & \int_\mbb{R} \left[ - \frac{\partial}{\partial y} \left(a(y, t) p_t(x, y)\right) + \frac{1}{2} \frac{\partial^2}{\partial y^2} \left( \sigma^2(y, t) p_t(x, y) \right) \right] f(y) \dif y,
\end{align*}
which  is true for all $f \in C^2(\mbb{R})$. Thus 
\begin{equation*}
    \frac{\partial p_t}{\partial t} = - \frac{\partial }{\partial y} \left(a p_t \right) + \frac{1}{2} \frac{\partial^2}{\partial y^2} \left(\sigma^2 p_t \right), 
\end{equation*}
which is called the \textbf{Fokker-Planck equation}. Regard $- \frac{\partial }{\partial y} \left(a p_t \right) + \frac{1}{2} \frac{\partial^2}{\partial y^2} \left(\sigma^2 p_t \right)$ as a function of $y$, and denote it as $\mathcal{L}^* p_t$.
\begin{definition}[The Fokker-Planck equation]
    The \textbf{Fokker-Planck equation} for a diffusion process is 
    \begin{equation*}
        \frac{\partial p_t}{\partial t} = - \frac{\partial }{\partial y} \left(a p_t \right) + \frac{1}{2} \frac{\partial^2}{\partial y^2} \left(\sigma^2 p_t \right).
    \end{equation*}
\end{definition}

Suppose the $a, \, \sigma$ are $t$-independent, then we get the stationary density 
\begin{equation*}
    p^*(x) = \frac{1}{Z} \exp\left( \int_0^x \frac{2 a(y) - (\sigma^2)'(y)}{\sigma^2(y)} \right) \dif y,
\end{equation*}
where $Z$ is the normalisation constant.
\begin{example}
    For an Orstein-Uhlenbeck process, 
    \begin{equation*}
        p*(x) = \frac{1}{Z} \exp \left( \int_0^x - \frac{2 \alpha y}{\sigma^2} \dif y \right) = \frac{1}{Z} \exp \left( - f\frac{\alpha x^2}{\sigma^2} \right), 
    \end{equation*}
    which is the density function of $\mathcal{N} (0, \frac{\sigma^2}{2\alpha})$.
\end{example}

\begin{proposition}
    The Fokker-Planck equation is an advection-diffusion equation with diffusion $D = \frac{\sigma^2}{2}$ and advection velocity $v = a - \sigma \sigma'$.
\end{proposition}

\begin{definition}[The Advection-Diffusion Equation]
    A general \textbf{advection-diffusion equation} for the density of a conserved quantity $\rho$ is 
    \begin{equation*}
        \frac{\partial \rho}{\partial t} + \div (\rho v - D \nabla \rho) = 0.
    \end{equation*}
\end{definition}

\begin{remark}
    For an advection-diffusion equation, the stationary density $\rho$ corresponds to $\div (\rho v - D \nabla \rho) = 0$, so in 1-D
    \begin{align*}
        \rho v = & D \nabla \rho \\ 
        \rho = & \frac{1}{Z} \exp \left( \int_0^x \frac{v(y)}{D(y)} \dif y \right).
    \end{align*}
\end{remark}

\begin{definition}[Real Brownian Motions]
    A \textbf{real Brownian motion} is better modelled by a Langevin equation:
    \begin{equation*}
        m \ddot{X} + \gamma \dot{X} = \sigma \xi.
    \end{equation*}
\end{definition}

\begin{remark}
    Note that this is an Ornstein-Uhlenbeck process for the velocity $\dot{X}$, so real Brownian motion is an integrated Ornstein-Uhlenbeck process. It is almost surely differentiabl in contrast to Brownian motion. But as Largvin noted the timescale for the mean reversion of $\gamma$ is about $10^{-8}$ seconds. As a result, if you look on timescales greater than $10^{-8}$ seconds, it looks like the Brownian motion 
    \begin{equation*}
        \gamma \dot{X} = \sigma \xi.
    \end{equation*}
\end{remark}
