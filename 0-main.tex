\documentclass[a4paper,12pt]{report}

% Math
\usepackage{amsmath}
\usepackage{amsthm}
\usepackage{thmtools}
\usepackage{amssymb}
\usepackage{calc}
\usepackage{bbm}
\usepackage{commath}
% Other packages
\usepackage{geometry}
\geometry{left=1.5cm,right=1.5cm,top=2cm,bottom=2cm} 
\usepackage{graphicx}
\usepackage{enumitem}
\usepackage{listings}
\usepackage{xcolor}

\definecolor{codegreen}{rgb}{0,0.6,0}
\definecolor{codegray}{rgb}{0.5,0.5,0.5}
\definecolor{codepurple}{rgb}{0.58,0,0.82}
\definecolor{backcolour}{rgb}{0.95,0.95,0.92}

\lstdefinestyle{mystyle}{
    backgroundcolor=\color{backcolour},   
    commentstyle=\color{codegreen},
    keywordstyle=\color{magenta},
    numberstyle=\tiny\color{codegray},
    stringstyle=\color{codepurple},
    basicstyle=\ttfamily\footnotesize,
    breakatwhitespace=false,         
    breaklines=true,                 
    captionpos=b,                    
    keepspaces=true,                 
    numbers=left,                    
    numbersep=5pt,                  
    showspaces=false,                
    showstringspaces=false,
    showtabs=false,                  
    tabsize=2
}

\lstset{style=mystyle}

\usepackage{hyperref}
\hypersetup{
    colorlinks=true,
    linkcolor=blue,
    filecolor=magenta,      
    urlcolor=cyan,
}

\urlstyle{same}

% \usepackage[
% backend=biber,
% style=alphabetic,
% sorting=ynt
% ]{biblatex}
% \addbibresource{ref.bib}
% \usepackage[nottoc]{tocbibind}

\newcommand{\mbb}[1]{\mathbb{#1}}
\newcommand{\mbf}[1]{\pmb{#1}}

\newcommand{\Prob}{\mathbb{P}}
\newcommand{\E}{\mathbb{E}}
\newcommand{\Var}{\mathrm{Var}}
\newcommand{\Cov}{\mathrm{Cov}}
\renewcommand{\div}{\text{div}}

\theoremstyle{plain}
\newtheorem{theorem}{Theorem}[section]

\theoremstyle{plain}
\newtheorem{proposition}{Proposition}[section]

\theoremstyle{definition}
\newtheorem{definition}{Definition}[section]

\theoremstyle{remark}
\newtheorem*{remark}{Remark}

\theoremstyle{definition}
\newtheorem{example}{Example}[section]

\title{Stochastic Modelling and Random Processes}
\author{Yiming MA}

\begin{document}
\maketitle
\tableofcontents

\chapter{Discrete-Time Markov Chains}

\section{Countable Discrete-Time Markov Chains}

One can extend much of what we have done for finite discrete-time Markov chains to the countably infinite case, e.g. the \textbf{simple random walk} on $\mbb{Z}$, but some results become more subtle. For example, the simple random walk is \textit{not SP-ergodic}, despite being \textit{irreducible}. Actually, it even \textit{fails to have a stationary probability}; also it is \textit{not aperiodic}, and it has a \textit{period} $2$.

\begin{example}
    Using definition of the simple random walk:

    \begin{equation*}
        Y_n = \sum_{i=0}^{n-1} X_i,
    \end{equation*}

    where $X_i$'s are independent and identically distributed, with 

    \begin{equation*}
        X_i = 
        \begin{cases}
            +1 \qquad & \text{with probability} \, p \\
            -1 \qquad & \text{with probability} \, 1-p
        \end{cases},
    \end{equation*}

    Compute the $\E[Y_n]$ and $\Var[Y_n]$.
\end{example}

One has to refine various concepts. 

\begin{definition}[The First Return Time]
    The \textbf{first return time} to state $x$ is defined as 
    
    \begin{equation*}
        T_x = \inf\{n \ge 1: X_n = x | X_0 = x\}.
    \end{equation*}
\end{definition}

\begin{remark}
    Notice that when the state space is finite and $x$ is recurrent, $T_x$ is finite. Since the state space here is countably infinite, $T_x$ is allowed to be infinte.
\end{remark}

\begin{definition}[Transience]
    Say $x \in S$ is \textbf{transient} if 
    
    \begin{equation*}
        \Prob[T_x = \infty] > 0.
    \end{equation*}
\end{definition}

\begin{remark}
    If $x \in S$ is transient, then with probability $1$ $X_n$ comes back to $x$ only finitely many times.
\end{remark}

\begin{definition}[Null Recurrence]
    Say $x \in S$ is \textbf{null recurrent} if 

    \begin{equation*}
        \Prob[T_x < \infty] = 1 \quad \text{and} \quad \E[T_x] = \infty.
    \end{equation*}
\end{definition}

\begin{definition}[Positive Recurrence]
    Say $x \in S$ is \textbf{positive recurrent} if 

    \begin{equation*}
        \Prob[T_x < \infty] = 1 \quad \text{and} \quad \E[T_x] < \infty.
    \end{equation*}
\end{definition}

\begin{remark}
    A communicating class is either \textbf{null recurrent}, which means every member is null recurrent, or \textbf{positive recurrent} which means every member is positive recurrent. 
\end{remark}

\begin{theorem}[Stationarity $\iff$ Positive Recurrence]
    An absorbing class has a stationary probability if and only if it is positive recurrent. Furthermore, if the class has one stationary probability, then it is uniquely determined by 
    
    \begin{equation*}
        \mbf{\pi}_x = \frac{1}{\E[T_x]}.
    \end{equation*}
\end{theorem}

\chapter{Continuous-Time Markov Chains}

\section{Continuous-Time Markov Chains}

We are now considering a continuous-time markov chain with a countable state space $S$ and the domain $T \in \mbb{R}$ (or $T \in \mbb{R}_+$), and we restrict $X: \mbb{R} \mapsto S$ to those which are \textit{piecewise constant} and \textit{right-continuous}, meaning

\begin{equation*}
    X(t) = 
    \begin{cases}
        \vdots &  \vdots \\ 
        s \quad & t \in [J_s, J_{s'}) \\ 
        s' \quad & t \in [J_{s'}, J_{s''}) \\ 
        \vdots & \vdots \\ 
    \end{cases}
\end{equation*}

\begin{definition}[Continuous-Time Markov Chains]
    $X(t): \mbb{R} \mapsto S$ is a \textbf{continuous-time Markov chain}, if it satisfies the \textbf{Markov property}

    \begin{equation*}
        \Prob[X(t_{n+1}) \in A | X(t_n) = s_n, \cdots, X(t_1) =  s_1] = \Prob[X(t_{n+1}) \in A | X(t_n) = s_n],
    \end{equation*}

    where $A \subset S$ and $t_1 < \cdots t_{n} < t_{n+1}$.
\end{definition}

\begin{definition}[Homogeneity]
    A continuous-time Markov chain is \textbf{homogeneous} if 

    \begin{equation*}
        \Prob [X(t+u) \in A | X(u) = s] = \Prob [X(t) \in A | X(0) = s].
    \end{equation*}
\end{definition}

\begin{remark}
    Homogeneity means time translation invariance.
\end{remark}


\begin{definition}[Transition Matrices]
    Let $(P_t)_{i,j} := \Prob [X(t) = j | X(0) = i]$, then $P_t$ is the transition matrix with time step $t$.
\end{definition}

\begin{remark}
    The $(i,j)$ element of the transition matrix $P_t$ can also be expressed as $P_t(i, j)$.
\end{remark}

\begin{theorem}[Chapman-Kolmogorov Equation]
    The transition matrix $P$ of a homogeneous Markov chain satisfies

    \begin{equation*}
        P_{t+u} = P_t P_u, \, P_0 = I.
    \end{equation*}

    \begin{proof}
        Notice that

        \begin{align*}
            (P_{t+u})_{i,j} = & \Prob[X(t+u) = j | X(0) = i] \\ 
            = & \sum_{k \in S} \Prob[X(t+u) = j | X(t) = k, \, X(0) = i] \Prob[X(t) = k | X(0) = i] \\ 
            = & \sum_{k \in S} \Prob[X(t+u) = j | X(t) = k] \Prob[X(t) = k | X(0) = i] \\ 
            = & \sum_{k \in S} \Prob[X(u) = j | X(0) = k] \Prob[X(t) = k | X(0) = i] \\ 
            = & \sum_{k \in S} (P_u)_{k, j} (P_t)_{i, k} \\ 
            = & (P_t)_{i, :} \; (P_u)_{:, j},
        \end{align*}

        where $(P_t)_{i, :}$ is the $i$-th row of $P_t$ and $(P_u)_{:, j}$ is the $j$-th column of $P_u$. Thus, $P_{t+u} = P_t P_u$. 

        And by definition, $(P_0)_{i,j} = \Prob[X_0 = j | X_0 = i] = \delta_{i,j}$, so $P_0 = I$.
    \end{proof}
\end{theorem}

\begin{definition}[Generator / Rate Matrix]
    Suppose $P_t$ is differentiable with respect to $t$ at $t = 0$, then 

    \begin{equation*}
        G :=  \left. \frac{\dif P_t}{\dif t} \right|_{t = 0}
    \end{equation*}
    is called the \textbf{generator} or the \textbf{rate matrix} of the process.
\end{definition}

\begin{proposition}
    $P_t = \exp(tG)$ in the sense of power series.
    
    \begin{proof}
        By the Chapman-Kolmogorov equation, we have 
        
        \begin{align*}
            P_{t+u} = & P_t P_u \\ 
            P_{t+u} - P_t = & P_t (P_u- I) \\ 
            \frac{P_{t+u} - P_t}{u} = & P_t \cdot \frac{P_u - I}{u} \\ 
            \lim_{u \to 0} \frac{P_{t+u} - P_t}{u} = & \lim_{u \to 0} P_t \cdot \frac{P_u - I}{u} \\ 
            \lim_{u \to 0} \frac{P_{t+u} - P_t}{u} = &  P_t \cdot \lim_{u \to 0} \frac{P_u - I}{u} \\ 
            \frac{\dif P_t}{\dif t} = &  P_t G,
        \end{align*}

        So $P_t = C \cdot \exp(tG)$, where $C$ is a constant diagonal matrix with diagonal elements being equal. By $P_0 = I$, we konw $C = I$.
    \end{proof}
\end{proposition}

\begin{proposition}
    The generator $G$ also satisfies 
    
    \begin{equation*}
        G \vec{1} = \vec{0}.
    \end{equation*}

    \begin{proof}
        For any probability distribution $\mbf{\pi}_t = \mbf{\pi}_0 P_t$ with initial distribution $\mbf{\pi}_0$, evolves by 

        \begin{equation*}
            \frac{\dif \mbf{\pi}_t}{\dif t} = \mbf{\pi}_0 \frac{\dif P_t}{\dif t} = \mbf{\pi}_0 P_t G = \mbf{\pi}_t G.
        \end{equation*}

        And by conservation of probability, we have $\mbf{\pi}_t \vec{1} = \vec{1}$, which implies $\mbf{\pi}_t G \vec{1} = \frac{\dif \mbf{\pi}_t \vec{1}}{\dif t} = 0$. Since $\mbf{\pi}_t$ is arbitrary, we have $G \vec{1} = 0$.
    \end{proof}
\end{proposition}

\begin{theorem}[The Master Equation]
    The equation 

    \begin{equation*}
        \frac{\dif \mbf{\pi}_t}{\dif t} = \mbf{\pi}_t G
    \end{equation*}
    
    can be written into

    \begin{equation*}
        \frac{\dif (\mbf{\pi}_t)_i}{\dif t} = \underbrace{\sum_{j \neq i} (\mbf{\pi}_t)_j G_{j, i}}_{\text{``gain''}}  - \underbrace{\sum_{j \neq i} (\mbf{\pi}_t)_i G_{i, j}}_{\text{``loss''}},
    \end{equation*}

    which is called the \textbf{master equation}.

    \begin{proof}
        For $i \neq j$, since $G_{i,j}$ is the rate at which the process goes from state $i$ to $j$, we have $G_{i,j} \ge 0$. By $G \vec{1} = \vec{0}$, we have 
    
        \begin{equation*}
            G_{i, i} = - \sum_{j \neq i} G_{i, j}.
        \end{equation*}
    
        So 
    
        \begin{align*}
            \frac{\dif (\mbf{\pi}_t)_i}{\dif t} = & \mbf{\pi}_t G_{:, i} \\ 
            = & \sum_{j \in S} (\mbf{\pi}_t)_j G_{j, i} \\ 
            = & \sum_{j \neq i} (\mbf{\pi}_t)_j G_{j, i} - \sum_{j \neq i} (\mbf{\pi}_t)_i G_{i, j}.
        \end{align*}
    \end{proof}
\end{theorem}

\begin{remark}
    The name ``master equation'' is exaggerated; it does not tell everything about the process, such as the correlations between states at different times.
\end{remark}

\begin{definition}[Stationarity]
    Say $\mbf{\pi} \in \Delta$ is \textbf{stationary} if $\mbf{\pi} G = 0$.
\end{definition}

\begin{definition}[Reversibility]
    Say $\mbf{\pi} \in \Delta$ is \textbf{reversible} if 

    \begin{equation*}
        \mbf{\pi}_i G_{i,j} = \mbf{\pi}_j G_{j, i}, \; \forall i,j \in S.
    \end{equation*}
\end{definition}

\begin{proposition}[$\text{Reversibility} \implies \text{Stationarity}$]
    If $\mbf{\pi} \in \Delta$ is reversibile, then it is also stationary.
\end{proposition}

\begin{proposition}
    $S$ is fintie $\implies$ $\exists$ stationary $\mbf{\pi}$.
\end{proposition}

There is an analogous decomposition of the state space $S$ into transient and recurrent states, and of the set of recurrent states into communicating components. And we have the same definition of an absorbing component. 

\begin{proposition}
    If $S$ is finite, then each absorbing component has a unique stationary probability $\mbf{\pi}$, and the space of starionary $\mbf{\pi}$ for the whole continuous-time Markov chain (up to normalisation) is the span of those for its absorbing components. Furthermore, $0$ is a semisimple eigenvalue of $G$.
\end{proposition}

\begin{theorem}
    Suppose $S$ is finite and $G$ has a unique absorbing component, then the process is SP-ergodic, which means 

    \begin{equation*}
        \lim_{t \to \infty} \mbf{\pi}_t = \mbf{\pi}_A, 
    \end{equation*}

    where $\mbf{\pi}_A$ is the stationary distribution of the absorbing component.
\end{theorem}

\begin{remark}
    Aperiodicity is automatic in continuous time.
\end{remark}

\section{Countable Continuous-Time Markov Chains}

Now suppose the state space $S$ of a continuous-time Markov chains is countable. We can define the null and positive recurrence as in the discrete-time case, but we have to find the return time differently. 

\begin{definition}[First Return Time]
    The \textbf{first return time} to state $x \in S$ is defined as 

    \begin{equation*}
        \inf\{t > J_1: X(t) = x\},
    \end{equation*}

    for $X(0) = x$. 
\end{definition}

\begin{proposition}
    Each positive recurrent absorbing component has a unique stationary probability distribution $\mbf{\pi}$, and 

    \begin{equation*}
        \mbf{\pi} = \frac{\E [W_x]}{\E[T_x]}.
    \end{equation*}
\end{proposition}

In continuous time, the process can get ``explosion''. 

\begin{definition}[Explosion]
    Let $J_\infty = \lim\limits_{n\to\infty} J_n$. If $\Prob [J_\infty = \infty] < 1$, then the continuous-time Markov chain is called \textbf{explosive}, which means there is a positive probability for infinitely many events in a bounded time.
\end{definition}

\begin{proposition}
    If $\sup\limits_{i \in S} |G_{i,i}| < \infty$, then the continuous-time Markov chain is \textit{not eplosive}.
\end{proposition}

\begin{example}[Explosion]
    Consider a birth and death process with  $X(0) = 1$, $\alpha_i = i^2$ and $\beta_i = 0$. Then 

    \begin{equation*}
        \E[J_\infty] = \sum_{i = 2}^\infty \E [W_i] = \sum_{i = 2}^\infty \frac{1}{\alpha_i} = \sum_{i = 2}^\infty \frac{1}{i^2} < \infty,
    \end{equation*}

    which means with probability $1$ $J_\infty$ is finite.
\end{example}

\section{Semi-Markov Chains}

\begin{definition}[Semi-Markov Chains]
    Take a discrete-time Markov chain and make a continuous-time process by waiting a time $W_x$ in each state $x \in S$ independently of previous and future states but not necessarily exponentially distributed.
\end{definition}

\begin{remark}
    Semi-Markov chains allow for latent periods and variations of infectivity with time from infection.
\end{remark}

\section{Gaussian Processes}

\begin{definition}[Gaussian Processes]
    Let $X: T \mapsto \mbb{R}$ be a stochasti process. $X(t)$ is called a \textbf{Gaussian process} if $\forall t_1, \cdots, t_n \in T$, $(X(t_1), \cdots, X(t_2))$ is a multivariate Gaussian random vector, i.e. it has the probability density function 

    \begin{equation*}
        f(x_1, \cdots x_n) = \frac{1}{\sqrt{(2\pi)^n |\Sigma|}} \exp \left( - \frac{1}{2} (\vec{x} - \vec{\mu}^T) \Sigma^{-1} (\vec{x} - \vec{\mu}) \right),
    \end{equation*}

    for some $\vec{\mu} = [\mu_1, \cdots, \mu_n]^T$ and some positive definite symmetric $n\times n$ matrix $\Sigma$.
\end{definition}

\begin{remark}
    A Guassian process is not necessarily Markov.
\end{remark}

\begin{proposition}
    There exist functions $m: T \mapsto \mbb{R}$ and $c:T\times T \mapsto \mbb{R}$ such that $\mu_i = m(t_i)$ and $\Sigma_{i,j} = c(t_i, t_j)$ with $c$ being ``positive definite'' i.e. such that $\Sigma$ is positive definite $\forall t_1, \cdots, t_n \in T$.
\end{proposition}

\begin{example}[Stationary Ornstein-Uhlenbeck Processes]
    Let $T = \mbb{R}$, $m(t) = 0$ and $c(t, t') = e^{-|t'-t|}$, then the process is called a \textbf{stationary Ornstein-Uhlenbeck process}.
\end{example}

One can allow degenerate Gaussians, e.g. Ornstein-Uhlenbeck with specified initial condition $X(0) = 0$, then $f(x_0) = \delta_0(x_0)$, which is not a Gaussian probability density function but can be viewed as the limit of a Gaussian density.

The best way to generate a Gaussian distribution is to use its characteristic function instead of its PDF.

\begin{definition}[Characteristic Functions]
    Let $\vec{X}$ be a random vector, then its characteristic function is 

    \begin{equation*}
        \phi(\vec{\theta}):=\E[e^{i \vec{\theta}^T\vec{X}}].
    \end{equation*}
\end{definition}

\begin{remark}
    For a multivariate Gaussian distribution with the mean vector $\vec{\mu}$ and covariance matrix $\mbf{\Sigma}$ (which is allowed to be positive semi-definite), its characteristic function is 
    
    \begin{equation*}
        \phi(\vec{\theta}) = e^{i\vec{\theta}^T \vec{\mu} - \frac{1}{2} \vec{\theta}^T \mbf{\Sigma} \vec{\theta}}.
    \end{equation*}
\end{remark}

We can include vector-valued Gaussian processes.

\begin{definition}[Multivariate Gaussian Processes]
    $\vec{X}:T \mapsto \mbb{R}^n$ is a \textbf{multivariate Gaussian process} if $X: T \times \{1, \cdots, n\} \mapsto \mbb{R}$ is a Gaussian process.
\end{definition}

\begin{definition}[Stationary Gaussian Processes]
    Suppose $T = \mbb{R} \times \mbb{K}$. The Gaussian process is \textbf{stationary}, if its mean function $m(t, k)$ is independent of $t$, and its covariance function $c(t, k; t', k')$ is dependent only on $t-t'$ and $k-k'$.
\end{definition}

\begin{remark}
    Gaussian processes are great for inference, because $\Prob [\text{parameters} | \text{data}]$ reduces to linear algebra.
\end{remark}

\section[Uncountable Markov Processes]{Markov Processes with $S = \mbb{R}$}

Suppose $X: T \mapsto \mbb{R}$, where $T$ can be $\mbb{Z}$ or $\mbb{R}$.

\begin{definition}[Markov Processes]
    $\{X(t): t \in T\}$ is a \textbf{Markov Processes} if it satisfies the Markov property

    \begin{equation*}
        \Prob [X(t_{n+1}) \in A | X(t_n) = s_n, \cdots, X(t_1) = s_1] = \Prob [X(t_{n+1}) \in A | X(t_n) = s_n],
    \end{equation*}

    where $A \subset \mbb{R}$ and $t_{n+1} > t_n > \cdots > t_1$.
\end{definition}

\begin{remark}
    There is a technical problem in the definition. The conditional probability is not well defined, since random variables $X_{t_n}, \cdots, X(t_1)$ now take values in $\mbb{R}$, and the probability that they take particular values is $0$. This will not be a problem if we restrict to any choice of interpretation of conditional probability such that 

    \begin{equation*}
        \Prob[X(t) \in A] = \int \Prob[X(t) \in A | X(0) = x] \dif \Prob [X(0) \le x] \qquad \text{(a Stieltjes integral)}.
    \end{equation*}
\end{remark}

\begin{definition}[Homogeneity]
    A Markov process is \textbf{homoegeneous} if 

    \begin{equation*}
        \Prob [X(t) \in A | X(t') = s] = \Prob [X(t-t') \in A | X(0) = s].
    \end{equation*}
\end{definition}

It is unlikely that $\Prob [X(t) = y | X(0) = x] > 0$, so instead we specify $\Prob [X(t) \in A | X(0) = x]$ for any measurable set $A \subset \mbb{R}$ as 

\begin{equation*}
    \int_A p_t(x,y) \dif y
\end{equation*}

for a transition density $p_t(\cdot, \cdot)$.

\begin{definition}[Transition Densities]
    A \textbf{transition probability} is a function $p_t(\cdot, \cdot): \mbb{R} \times \mbb{R} \mapsto \mbb{R}$ such that 

    \begin{equation*}
        \Prob[X(t) \in A | X(0) = x] = \int_A p_t(x,y) \dif y
    \end{equation*}
\end{definition}

\begin{theorem}[The Chapman-Kolmogorov Equation]
    The Markov property and homogeneity implies the \textbf{Chapman-Kolmogorov equation} 

    \begin{equation*}
        p_{t+u}(x, y) = \int_\mbb{R} p_t(x, z) p_u(z, y) \dif z.
    \end{equation*}
\end{theorem}

\subsection{Jump Processes}

\begin{definition}[Jump Processes]
    $\{X(t): t \in \}$ is a \textbf{jump process} if 
    
    \begin{itemize}
        \item there is a \textbf{jump rate density} $r(x,y)$ with the \textbf{exit rate} 

        \begin{equation*}
            R(x) = \int_\mbb{R} r(x,y) \dif y \le M < \infty, \; \forall x \in \mbb{R},
        \end{equation*}
    
        where $M \in \mbb{R}$ is a constant;

        \item its transition density satisfies 
        
        \begin{equation*}
            p_{\Delta t}(x, y) = r(x, y) \Delta t + \left( 1 - R(x) \Delta t \right) \delta(y - x) + o(\Delta t), \; \text{as} \, \Delta t \to 0.
        \end{equation*}
    \end{itemize}
\end{definition}

\begin{theorem}[The Kolmogorov-Feller Equation]
    The Chapman-Kolmogorov equation of a jump process turns into the \textbf{Kolmogorov-Feller equation} for initial condition $x \in \mbb{R}$

    \begin{equation*}
        \frac{\partial}{\partial t} p_t(x, y) = \int_\mbb{R} p_t(x, z) r(z, y) - p_t(x, y) r(y, z)\dif z
    \end{equation*}
\end{theorem}

\subsection{Diffusion Processes}

\begin{definition}[The Brownian Motion]
    The \textbf{Brownian motion} is a Gaussian process $B: \mbb{R}_+ \mapsto \mbb{R}$ with $m(t) = 0$ and $c(t, t') = \min (t, t')$ and almost surely continuous paths.
\end{definition}

\begin{proposition}[Brownian Motions are Markov]
    A Brownian motion is Markov, and it has independent increments: $\forall t_1 < \cdots < t_n$, $(X(t_{k+1}) - X_{t_k})_{k = 1, \cdots, n-1}$ are independent variables. 
\end{proposition}

\begin{proposition}[Brownian Motions are Homeogeneous]
    Furthermore, the increments are stationary: $X(t) - X(s)$ and $X(t-s) - X(0) = X(t-s)$ have the same distribution, for $t \> s$. So $B(t)$ is homoegeneous.
\end{proposition}

\begin{remark}
    $B(t)$ is not stationary.
\end{remark}

\begin{proposition}
    The transition density $p_t(x,y)$ of a Brownian motion is a Gaussian PDF with mean $y-x$ and variance $t$, which satisfies the heat equation (or diffusion equation): 

    \begin{equation*}
        \frac{\partial p_t}{\partial t} = \frac{1}{2} \frac{\partial^2 p_t}{\partial y^2}
    \end{equation*}

    with the initial condition $p_0(x, y) = \delta(y - x)$.
\end{proposition}

\begin{proposition}
    Brownian motions are normally distributed: $B(t) \sim \mathcal{N}(0, t)$.
\end{proposition}

\begin{proposition}
    $B(t)$ is scale-invariant: $B(\lambda t)$ and $\sqrt{\lambda} B(t)$ have the same distribution. 
\end{proposition}

\begin{proposition}
    $B(t)$ is almost surely continuous, but it is also almost surely nowhere differentiable. Actually, 

    \begin{equation*}
        \xi_{t, h} := \frac{B(t+h) - B(t)}{h} \sim \mathcal{N}\left(0, \frac{1}{h}\right).
    \end{equation*}
\end{proposition}

Although Brownian motions are almost surely nowhere differentiable, we can still informally talk about the limit proecss $\xi_t := \lim\limits_{h \to 0} \xi_{t, h}$. 

\begin{definition}[Gaussian White Noises]
    $\xi_t := \lim\limits_{h \to 0} \xi_{t, h}$ is called the \textbf{Gaussian white noise}.
\end{definition}

\begin{remark}
    The Gaussian white noise can be considered as a limiting case of a Gaussian process with mean $m(t) = 0$ and $c(t, t') = \delta(t - t')$.
\end{remark}

\begin{proposition}
    $B(t) = \int_0^t \xi_{t'} \dif t'$, or we can write it as a stochastic differential equation 

    \begin{equation*}
        \frac{\dif B}{\dif t} = \xi, 
    \end{equation*}

    with $B(0) = 0$.
\end{proposition}

\section{Generators as Operators}

\subsection{Generators of Discrete Continuous-Time Markov Chains}

For a continuous-time Markov chain with a countable state space $S$, for any function $f: S \mapsto \mbb{R}$, we have 

\begin{equation*}
    \E [f(X(t))] = \sum_{x \in S} \mbf{\pi}_t(x) f(x) = \mbf{\pi}_t \vec{f},
\end{equation*}

where $\vec{f}$ is a column vector of values of $f$ at all the state $x \in S$.

We may be interested in how fast $\E [f(X(t))]$ varies with time $t$, so 

\begin{equation*}
    \frac{\dif}{\dif t} \E [f(X(t))] = \frac{\dif}{\dif t}\mbf{\pi}_t \vec{f} = \mbf{\pi}_t G \vec{f}.
\end{equation*}

Thus, we can think of the generator $G$ as acting on the function $f$ by 

\begin{equation*}
    (Gf) (x) = \sum_{y \in S} G_{x, y} f(y) = \sum_{\substack{y \neq x \\ y \in S}} G_{x, y} (f(y) - f(x)).
\end{equation*}

\subsection{Generators of Continuous Continuous-Time Markov Chains}

The idea of generators as operators can be extended to $S = \mbb{R}$ by replacing matrices and vectors with operators and functions.

\subsubsection{Generators of Brownian Motions}
 For a Brownian motion,

\begin{align*}
    \frac{\dif}{\dif t} \E [f(X(t))] = & \frac{\partial}{\partial t} \int_\mbb{R} p_t(x, y) f(y) \dif y \\ 
    = & \int_\mbb{R} \frac{\partial}{\partial t} p_t(x, y) f(y) \dif y \\ 
    = & \frac{1}{2} \int_\mbb{R} \frac{\partial^2}{\partial y^2} p_t(x, y) f(y) \dif y \\ 
    = & \E [(\mathcal{L}f)(X(t))]
\end{align*}

with $(\mathcal{L}f)(x) := \frac{1}{2} f''(x)$, assuming $f$ is twice differentiable and $f(x) \, \& f'(x) \to 0$ as $x \to \pm \infty$ (integration by parts). $\mathcal{L}$ is the generator but now a linear operator on functions.

\subsubsection{Generators of Jump Processes}

For a jump process on $\mbb{R}$, 

\begin{equation*}
    (\mathcal{L}f)(x) = \int_\mbb{R} r(x,y) [f(y) - f(x)] \dif y.
\end{equation*}

We can obtain the Brownian motion as a scaling limit of a jump process. Take a jump process $X(t)$ with $r(x, y) = q(y-x)$ such that $\int_\mbb{R} z q(z) \dif z = 0$ and $\int_\mbb{R} z^2 q(z) \dif z = \sigma^2 \in (0, \infty)$. Then $\forall T > 0$, with $X(0) = 0$,

\begin{equation*}
    \left. \frac{\epsilon}{\sigma} X(\frac{t}{\epsilon^2}) \right|_{t \in [0, T]} \xrightarrow{\text{d}} \left. B(t) \right|_{t \in [0, T]}, \; \text{as} \; \epsilon \to 0.
\end{equation*}

We can prove this by Taylor expansion of the generator 

\begin{equation*}
    f(y) = f(x) + (y-x) f'(x) + \frac{1}{2} (y-x)^2 f''(x) + \cdots,
\end{equation*}

and tightness of the set $S$ of probability distributions for the scaled jump process: $\forall \eta > 0$, $\exists K \in \mbb{R}$ such that for all $\mu \in \bar{S}$, $\mu (K^c) < \eta$.

\end{document}