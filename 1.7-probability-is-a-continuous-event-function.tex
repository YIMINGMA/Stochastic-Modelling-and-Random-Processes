\section{Probability Is a Continuous Event Function}

\begin{definition}[Increasing Sequences]
    We say that the sequence of events $A_1, A_2, \cdots$ is an \textbf{increasing sequence} if $A_n \subset A_{n+1}$ for all $n \ge 1$. 
\end{definition}

\begin{definition}[Limits of Increasing Sequences]
    If $A_n, n \ge 1$ is an increasing sequence of events, we define its limit by 
    \begin{equation*}
        \lim_{n \to \infty} A_n = \bigcup_{i=1}^\infty A_i.
    \end{equation*}
\end{definition}

\begin{definition}[Decreasing Sequences]
    We say that the sequence of events $A_1, A_2, \cdots$ is an \textbf{decreasing sequence} if $A_n \supset A_{n+1}$ for all $n \ge 1$. 
\end{definition}

\begin{definition}[Limits of Decreasing Sequences]
    If $A_n, n \ge 1$ is an decreasing sequence of events, we define its limit by 
    \begin{equation*}
        \lim_{n \to \infty} A_n = \bigcap_{i=1}^\infty A_i.
    \end{equation*}
\end{definition}

We now show that probability is a continuous event function.

\begin{proposition}
    If $A_n$, $n \ge 1$ is either an increasing or a decreasing sequence of events, then
    \begin{equation*}
        \Prob(\lim\limits_{n\to\infty} A_n) = \lim\limits_{n\to\infty} \Prob (A_n).
    \end{equation*}

    \begin{proof}
        Suppose $A_n$, $n \ge 1$ is an increasing sequence of events. Now, define the events $B_n$, $n \ge 1$, by letting $B_n$ be the set of points that are in $A_n$ but were not in any of the events $A_1, \cdots, A_{n-1}$. That is, we let $B_1 = A_1$, and for $n > 1$ let 
        \begin{align*}
            B_n = & A_n \bigcap \left(\bigcup_{i=1}^{n-1} A_i\right)^c \\ 
            = & A_n A_{n-1}^c
        \end{align*}
        where the final equality used that $A_1, A_2, \cdots$ being increasing implies that $\cup_{i=1}^{n-1} = A_{n-1}$. It is easy to see that the events $B_n$, $n \ge 1$ are mutually exclusive, and are such that 
        \begin{equation*}
            \cup_{i=1}^n B_i = \cup_{i=1}^n A_i = A_n, \; n \ge 1
        \end{equation*}
        and 
        \begin{equation*}
            \cup_{i=1}^\infty B_i = \cup_{i=1}^\infty A_i.
        \end{equation*}
        Hence, 
        \begin{align*}
            \Prob (\lim\limits_{n\to\infty} A_n) = & \Prob (\cup_{i=1}^\infty A_i) \\ 
            = & \Prob (\cup_{i=1}^\infty B_i) \\ 
            = & \sum_{i=1}^\infty \Prob (B_i) \quad (\text{since $B_i$'s are mutually exclusive}) \\ 
            = & \lim\limits_{n\to\infty} \sum_{i=1}^n \Prob (B_i) \\ 
            = & \lim\limits_{n\to\infty} \Prob (\cup_{i=1}^n B_i) \\ 
            = & \lim\limits_{n\to\infty} \Prob (\cup_{i=1}^n A_i) \\ 
            = & \lim\limits_{n\to\infty} \Prob (A_n).
        \end{align*}

        Now suppose $A_n$ is decreasing, then $A_n^c$ is increasing. So we have 
        \begin{align*}
            \Prob (\lim\limits_{n\to\infty} A_n) = & \Prob (\cap_{n=1}^\infty A_n) \\ 
            = & \Prob \left( \left( \cup_{n=1}^\infty A_n^c \right)^c \right) \\ 
            = & 1 - \Prob \left( \left( \cup_{n=1}^\infty A_n^c \right) \right) \\ 
            = & 1 - \lim\limits_{n\to\infty} \Prob (A_n^c) \\ 
            = & \lim\limits_{n\to\infty} (1 - \Prob (A_n^c)) \\ 
            = & \lim\limits_{n\to\infty} \Prob (A_n).
        \end{align*}
    \end{proof}
\end{proposition}

\begin{example}
    Consider a population of individuals and let all individuals initially
    present constitute the first generation. Let the second generation consist of all offspring of the first generation, and in general let the $(n + 1)$st generation consist of all the offspring of individuals of the $n$th generation. Let $A_n$ denote the event that there are no individuals in the $n$th generation. Because $A_n \subset A_{n+1}$ it follows that $\lim\limits_{n\to\infty} A_n = \cup_{i=1}^\infty A_i$. Because $\cup_{i=1}^\infty A_i$ is the event that the population eventually dies out, it follows from the continuity property of probability that
    \begin{equation*}
        \lim_{n\to\infty} \Prob (A_n) = \Prob (\text{population dies out}).
    \end{equation*}
\end{example}