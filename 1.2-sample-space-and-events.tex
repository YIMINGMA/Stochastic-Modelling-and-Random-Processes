\section{Sample Space and Events}

uppose that we are about to perform an experiment whose outcome is not predictable in advance. However, while the outcome of the experiment will not be known in advance, let us suppose that the set of all possible outcomes is known. 

\begin{definition}[Sample Spaces]
    This set of all possible outcomes of an experiment is konwn as the \textbf{sample space} of the experiment and is denoted by $\SS$.
\end{definition}

Some examples are the following.
\begin{enumerate}
    \item[$\mbf{1}$.] If the experiment consits of the flipping of a coin, then
    \begin{equation*}
        \SS = \{H, T\}
    \end{equation*}
    where $H$ means that the outcome of the toss is head and $T$ that it is a tail.
    \item[$\mbf{2}$.] If the experiment consists of  rolling a die, then the sample space is 
    \begin{equation*}
        \SS = \{1,2,3,4,5,6\}
    \end{equation*}
    where the outcome $i$ means that $i$ appeared on the die, $i = 1,2,3,4,5,6$.
    \item[$\mbf{3}$.] If the experiment consists of flipping two coins, then the sample sapce consists of the following four points:
    \begin{equation*}
        \SS = \{(H,H), (H,T), (T, H), (T,T)\}.
    \end{equation*}
    The outcome will be $(H,H)$ if both coins come up heads; it will be $(H,T)$ if the first coin comes up heads and the second comes up tails; it will be $(T,H)$ if the first comes up tails and the second heads; and it will be $(T,T)$ if both coins come up tails.
    \item[$\mbf{4}$.] If the experiment consists of rolling two dice, then the sample space consists of the following $36$ points:
    \begin{equation*}
        \SS = \begin{bmatrix}
            (1,1) & (1,2) & (1,3) & (1,4) & (1,5) & (1,6) \\ 
            (2,1) & (2,2) & (2,3) & (2,4) & (2,5) & (2,6) \\ 
            (3,1) & (3,2) & (3,3) & (3,4) & (3,5) & (3,6) \\ 
            (4,1) & (4,2) & (4,3) & (4,4) & (4,5) & (4,6) \\ 
            (5,1) & (5,2) & (5,3) & (5,4) & (5,5) & (5,6) \\ 
            (6,1) & (6,2) & (6,3) & (6,4) & (6,5) & (6,6) \\ 
        \end{bmatrix}
    \end{equation*}
    where the outcome $(i,j)$ is said to occur if $i$ appears on the first die and $j$ on the second die. 
    \item[$\mbf{5}$.] If the experiment consists of measuring the lifetime of a car, then the sample space consists of all nonnegative real numbers. That is, 
    \begin{equation*}
        \SS = [0, \infty).
    \end{equation*}
\end{enumerate}

\begin{definition}[Events]
    Any subset $E$ of the sample space $\SS$ is konwn as an event. 
\end{definition}

Some examples of events are the following.

\begin{enumerate}
    \item[$\mbf{1'}$.] In Example (1), if $E = \{H\}$, then $E$ is the event that a head appears on the flip of the coin. Similarly, if $E = \{T\}$, then $E$ would be the event that a tail appears.
    \item[$\mbf{2'}$.] In Example (2), if $E = \{1\}$, then $E$ is the event thhat one apperas on the roll of the die. If $E = \{2,4,6\}$, then $E$ would be the event that an even number appears on the roll.
    \item[$\mbf{3'}$.] In Example (3), if $E = \{(H,H), (H,T)\}$, then $E$ is the event that a head appears on the first coin. 
    \item[$\mbf{4'}$.] In Example (4), if $E = \{(1,6), (2,5), (3,4), (4,3), (5,2), (6,1)\}$, then $E$ is the event that the sum of the dice equals seven. 
    \item[$\mbf{5'}$.] In Example (5), if $E = (2,6)$, then $E$ is the event that the car lasts between two and six years. 
\end{enumerate}

We say that event $E$ occurs when the outcome of the experiment lies in $E$. For any two events $E$ and $F$ of a sample space $\SS$ we define the new event $E \cup F$ to consist of all outcomes that are either in $E$ or in $F$ or in both $E$ and $F$. That is, the event $E \cup F$ will occur if either $E$ or $F$ occurs. For example, in (1) if $E = \{H\}$ and $F = \{T\}$, then 
\begin{equation*}
    E \cup F = \{H,T\}.
\end{equation*}
That is, $E \cup F$ would be the whole sample space $\SS$. In (2) if $E = \{1, 3, 5\}$ and $F = \{1,2,3\}$, then 
\begin{equation*}
    E \cup F = \{1,2,3,5\},
\end{equation*}
and thus $E \cup F$ would occur if the outcome of the die is $1$ or $2$ or $3$ or $5$.


\begin{definition}[Unions]
    The event $E \cup F$ is often referred to as the \textbf{union} of the event $E$ and the event $F$.
\end{definition}

\begin{definition}[Intersections]
    For any two events $E$ and $F$, we may also define the new event $EF$, sometimes written $E \cap F$, and reffered to as the \textbf{intersection} of $E$ and $F$.
\end{definition}

As follows, $EF$ consists of all outcomes which are both in $E$ and $F$. That is, the event $EF$ will occur only if both $E$ and $F$ occur. In Example (2), if $E = \{1,3,5\}$ and $F = \{1,2,3\}$, then
\begin{equation*}
    EF = \{1,3\},
\end{equation*}
and thus $EF$ occur if the outcome of the die is either $1$ or $3$. In Example (1), if $E = \{H\}$ and $F = \{T\}$, then the event $EF$ would not consist of any outcomes and hence could not occur.

\begin{definition}[The Null Event]
    To give such an event a name, we shall refer to it as the \textbf{null event} and denote it by $\emptyset$. (That is, $\emptyset$ refers to the event consisting of no outcomes.) 
\end{definition}

\begin{definition}[Mutual Exclusivity]
    If $EF = \emptyset$, then $E$ and $F$ are said to be \textbf{mutually exclusive}.
\end{definition}

We also define unions and intersections of more than two events in a similar manner. If $E_1, E_2, \cdots$ are events, then the union of these events, denoted by $\bigcup_{n = 1}^\infty E_n$, is defined to be the event that consists of all outcomes that are in $E_n$ for at least one value of $n = 1, 2, \cdots$. Similarly, the intersection of the events $E_n$, denoted by $\bigcap_{n = 1}^\infty E_n$, is defined to be the event consisting of thoese outcomes that are in all of the events $E_n$, $n = 1,2, \cdots$.

\begin{definition}[The Complement]
    Finally, for any event $E$ we define the new event $E^c$, referred to as the \textbf{complement} of $E$, to consist of all outcomes in the sample space $\SS$ that are not in $E$. That is,  $E^c$ will occur if and only if $E$ does not occur. 
\end{definition}

In Example (4) if $E = \{(1,6), (2,5), (3,4), (4,3), (5,2), (6,1)\}$, the $E^c$ will occur if the sum of the dice does not equal seven. 

\begin{remark}
    Also note that since the experiment must result in some outcome, it follows that $\SS^c = \emptyset$.
\end{remark}
